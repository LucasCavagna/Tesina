\chapter{Conclusiones finales y trabajos futuros}

\section*{Conclusiones finales}

Durante el desarrollo de esta tesina se obtuvieron los siguientes resultados:
\begin{itemize}
    \item Se logró con éxito construir una implementación de conjuntos ordenados capaz de manejar múltiples dimensiones y paso igual o distinto a uno.  
    Además, esta presenta un rendimiento superior al de la implementación de conjuntos desordenados, especialmente cuando se utilizan más de una dimensión como se vio en los casos de prueba sintéticos y en los casos de prueba.  
    
    Sin embargo, en comparación con la implementación de conjuntos ordenados densos, el rendimiento de la presente implementación en los casos de prueba sintéticos con una única dimensión y paso igual a uno resulta notablemente inferior en ciertos casos, como en las operaciones de complemento o unión, mientras que, por ejemplo, en la intersección el rendimiento es muy similar.

    \item Se consiguió desarrollar una implementación de \textit{piecewise maps} ordenados, 
    constituyendo así la primera implementación de \textit{piecewise maps} que aprovecha el orden 
    para mejorar su desempeño.  
    
    Además, los resultados de los casos de prueba muestran que esta implementación presenta 
    un rendimiento superior al de los \textit{piecewise maps} desordenados, 
    aunque la mejora no es tan marcada como la observada en la implementación de conjuntos ordenados 
    frente a los conjuntos desordenados.  
    Cabe destacar que las pruebas se realizaron utilizando los
    \textit{piecewise maps} desordenados optimizados, y no la versión original con la que se comenzó el desarrollo de esta tesina.
\end{itemize}

Por ende, en conclusión, se lograron cumplir los objetivos planteados para esta tesina, 
incluso alcanzando las mejoras de rendimiento esperadas.  
No obstante, hacia el final del desarrollo ciertas cuestiones, la cuales, resultan lo suficientemente relevantes como para ser enunciadas, pueden utilizarse para el desarrollo de nuevos trabajos en el futuro.  
Estas serán ampliadas en la siguiente sección dedicada a los trabajos futuros.

\section*{Trabajos futuros}

En este trabajo se presentaron, en un capítulo propio, un conjunto de optimizaciones 
aplicadas a ciertas operaciones de los \textit{piecewise maps} desordenados.  
Estas surgieron durante el desarrollo de la implementación ordenada, 
ya que se partió de la versión desordenada para construir esta última.  
Sin embargo, estas no son las únicas optimizaciones que pueden realizarse sobre 
los \textit{piecewise maps} desordenados, 
ni significa que, por no haberse presentado ninguna para los conjuntos desordenados, no existan optimizaciones posibles para estos últimos.

En particular, los diferentes criterios de solapamiento enunciados a lo largo de este trabajo 
no requieren del orden para funcionar correctamente.  
Por ende, los criterios aplicados en las operaciones de conjuntos ordenados 
también podrían emplearse en conjuntos desordenados; 
de igual forma, en el caso de los \textit{piecewise maps}, 
los criterios definidos para los ordenados podrían aplicarse a los desordenados.

Cabe destacar que, además de utilizar estos criterios en las implementaciones desordenadas, 
también sería posible trasladarlos a estructuras de jerarquía inferior.  
Es decir, el criterio de solapamiento utilizado en conjuntos podría desplazarse 
a la implementación de multi-intervalos; 
y, de manera análoga, los utilizados a nivel de \textit{piecewise maps} 
podrían trasladarse a conjuntos.

No obstante, esta migración debería realizarse con cautela, 
con el objetivo de no disminuir el rendimiento de ninguna de las implementaciones que harían uso de estos criterios.
